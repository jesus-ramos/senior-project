\documentclass{beamer}

\usetheme{CambridgeUS}
\usecolortheme{beaver}

\usepackage{adjustbox}
\usepackage{algorithmicx}
\usepackage{algorithm}
\usepackage{algpseudocode}
\usepackage{color}
\usepackage{listings}

\definecolor{dkgreen}{RGB}{5,93,16}

\lstset{
  language=C,
  keywordstyle=\color{blue},
  commentstyle=\color{dkgreen}
}

\title[Cache Management]{
  Implementation of Cache Management Algorithms in DM-Cache
}
\author{Jesus Ramos}
\institute[FIU]{Florida International University}
\date{}

\begin{document}

\maketitle

\section{Background}

\subsection{Implementation Details}

\begin{frame}
  \frametitle{DM-Cache Current Implementation}

  \begin{itemize}
    \item Uses a hash table of fixed size which is initialized during cache
    creation
    \item Eviction is done when a hash collision is detected, simply evict the
    previous block from the entry in the hash table
  \end{itemize}

  Drawbacks:
  \begin{itemize}
    \item Size of cache has to be power of 2 due to hashing function used
    \item Hash table occupies a lot of memory in RAM
    \item Evictions have no notion of block use or frequency of access
    \item Possibility of ``wasteful'' entries in table (empty entries)
  \end{itemize}

\end{frame}

\begin{frame}
  \frametitle{New Implementation}

  \begin{itemize}
    \item Replaces hash table with Radix Tree indexed by disk sectors
    \item Adds linked list to manage LRU/LFU eviction policies
    \item Uses radix tree to index entries in the list quickly
  \end{itemize}

  Significance of changes:
  \begin{itemize}
    \item Indexing time for Radix Tree $O(logn) >$ Index time for hash table $O(1)$
    \item Radix Tree only occupies as much memory as there are blocks in the
    cache and doesn't have the drawbacks of sparse hash tables
    \item Eviction is now done based on recency or frequency of use rather than
    arbitrary hash collisions
  \end{itemize}

\end{frame}

\begin{frame}
  \frametitle{Algorithm - LRU}

  \begin{algorithmic}
  \State $block \gets$ look-up requested block in $radix\_tree$
  \If {$block$ is not cached}
    \If {$cache$ is full}
      \State evict from tail of $lru\_list$
    \EndIf
    \State initialize new $block$
    \State insert $block$ into $radix\_tree$
  \EndIf
  \State move $block$ to head of $lru\_list$
\end{algorithmic}

\end{frame}

\begin{frame}
  \frametitle{Algorithm - LFU}

  \begin{algorithmic}
  \State $block \gets$ look-up requested block in $radix\_tree$
  \If {$block$ is not cached}
    \If {$cache$ is full}
      \State evict from tail of $lfu\_list$
    \EndIf
    \State initialize new $block$
    \State insert $block$ into $radix\_tree$
    \State insert $block$ at tail of $lfu\_list$ with $hit\_count$ of $1$
  \Else
    \State $block.hit\_count \gets block.hit\_count + 1$
    \State $next\_block \gets$ adjacent entry of $block$ from $lfu\_list$
    \If {$block.hit\_count \geq next\_block.hit\_count$}
      \State swap $block$ and $next\_block$ in place
    \EndIf
  \EndIf
\end{algorithmic}


\end{frame}

\begin{frame}
  \frametitle{Modifications - Cache Struct}

  Original definition for cache structure:
  \begin{adjustbox}{width=\textwidth,keepaspectratio}
    \lstinputlisting{../snippets/cache_struct.c}
  \end{adjustbox}

  \vspace{\baselineskip}

  New definition for cache structure:
  \begin{adjustbox}{width=\textwidth,keepaspectratio}
    \lstinputlisting{../snippets/cache_struct_new.c}
  \end{adjustbox}

\end{frame}

\begin{frame}
  \frametitle{Modifications - Cacheblock Struct}

  Additions to cacheblock structure for LFU:
  \begin{adjustbox}{width=\textwidth,keepaspectratio}
    \lstinputlisting{../snippets/new_cacheblock_struct.c}
  \end{adjustbox}

\end{frame}

\begin{frame}
  \frametitle{Modifications - Cache Hit}

  \begin{adjustbox}{width=\textwidth,keepaspectratio}
    \lstinputlisting{../snippets/new_cache_hit.c}
  \end{adjustbox}

\end{frame}

\begin{frame}
  \frametitle{Modifications - Cache Miss}

  \begin{adjustbox}{width=\textwidth,keepaspectratio}
    \lstinputlisting{../snippets/new_cache_miss.c}
  \end{adjustbox}

\end{frame}

\begin{frame}
  \frametitle{Modifications - Cache Eviction}

  \begin{adjustbox}{width=\textwidth,keepaspectratio}
    \lstinputlisting{../snippets/new_cache_invalidate.c}
  \end{adjustbox}

\end{frame}

\end{document}
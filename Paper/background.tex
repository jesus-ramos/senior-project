\section{Background}

\subsection{DM-Cache}

DM-Cache is a block-level caching mechanism implemented as a module in the Linux
kernel \cite{DM-Cache}. It allows for specification of a local cache device
where recently used blocks can be stored for faster access time. It supports
both write-back and write-through functionality for handling cached data.

The purpose of DM-Cache is to cache recently used blocks on local storage
instead of retrieving them from some external storage source. This helps to
alleviate network bandwidth issues as well as reduce response times and
indirectly save power by issuing less requests to the storage node.

The current implementation of DM-Cache uses a hash table that maps logical block
addresses to entries within the hash table and evictions are done when two block
addresses map to the same entry but contain different data. This eviction scheme
does not take into account most recently used blocks and it also does not make
any attempt to pre-fetch any other blocks.

\section{Design and Implementation}
\label{sec:implementation}

\subsection{Modifications to DM-Cache}

\subsubsection{Radix Tree}

DM-Cache currently employs a hash table for managing the cached blocks. Eviction
of cached blocks is done when there is a conflict in the hash table where two
blocks map to the same entry. Cache sizes are also restricted to powers of two
in size because of the hash function used in the implementation.

In order to improve cache management and remove the cache size restrictions the
hash table was replaced with a radix tree. This allowed for implementation of
the LRU (Least Recently Used) and LFU (Least Frequently Used) cache management
algorithms.

\section{Introduction}

Data center power consumption is growing very rapidly. In 2011 it was estimated
that U.S. data centers consumed about 100 billion kWh of electricity at a
staggering cost of \$7.4 billion \cite{data-center-power}. Because data centers
house so many computers even small savings per machine can lead to large
savings overall.

Power consumption of an SSD when compared to standard hard disks is much lower
due to the lack of mechanical parts. This makes them ideal for saving power and
increasing performance in data centers. Storage cost of SSD's when compared to
traditional hard drives is also much higher so using them as primary storage can
become very expensive and impractical. Work such as that by Lee et al.\ has
shown that SSD's can be used sparingly to improve throughput of enterprise
database applications \cite{enterprise-ssd}.

We can leverage SSD's in data centers not only to improve throughput and
performance of applications, but to save energy as well. Using an SSD as a local
cache device can help mitigate the cost of having to read from a mechanical disk
and provides another caching layer between physical memory and the hard
disk. Energy is saved not only by issuing less requests to the mechanical disk
but also from the increased performance granted by the SSD. A workload can
finish in much less time if requests can be satisfied from faster storage which
means the workload will consume less energy as energy usage scales with the
amount of time the workload ran for.

From previous work by Mesiner et al.\ it was shown that many data center
workloads are often bursty with frequent periods of short idle time
\cite{powernap}. By increasing the performance of certain workloads these idle
periods can be made more frequent and allow for better exploitation of power
saving policies. If these idle periods become long enough then resources can be
consolidated onto more idle machines allowing for even better energy savings.

The remainder of this paper is organized as follows. In the following section we
provide background information on the use of SSD's as local cache
devices. Section ~\ref{sec:related} discusses related work and current
approaches used to save energy in cloud based systems as well as different
approaches to caching. Section ~\ref{sec:implementation} details the design and
implementation of the caching policies that were implemented as well as the
tools used to measure power consumption online. In Section ~\ref{sec:evaluation}
we present our test setup and evaluation of these caching algorithms as well as
power consumption statistics for each. Our discussion and future work is
detailed in Section ~\ref{sec:discussion} and Section ~\ref{sec:future}
respectively and we conclude in Section ~\ref{sec:conclusion}.

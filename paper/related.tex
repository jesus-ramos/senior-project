\section{Related Work}
\label{sec:related}

Caching data on solid state drives (SSD) in order to save power is not a new
concept. Some of the ideas presented in this paper are borrowed from previous
work on using SSD's as cache for the slower and more energy consuming hard disk
drives (HDD) \cite{flash-disk-hybrid}. In the work of Useche et al., the use of
an external caching system with enabled pre-fetching, caching and buffering of
data is explored for the purpose of reducing disk activities and saving power
\cite{extern-cache-energy-saving}. The following sections of this related works
discuss some of the power measuring papers that we explored, as well as the
policies that we will use in our system.

\subsection{Power Measuring}

Methods for accurately modeling power consumption in HDD's have been extensively
explored. Zedlewski et al. developed a modeling technique for empirically
extracting parameters from hard disk drives, taking seeks, spins, and idle power
into consideration \cite{hard-disk-power}. Although simulators have been created
for evaluating performance on NAND Flash-based SSD's \cite{flashsim}, to the best
of our knowledge, there is little development on accurate modeling of energy
consumption for solid state drives. One source analyzes the energy efficiency of
flash-based SSD's, but no advanced modeling technique is used
\cite{ssd-energy-efficiency}. For the purpose of this paper, we use simple power
monitoring tools to measure instantaneous power consumption.

\subsection{Cache Partitioning and Replication}

Previous work on replication shows how a simple key-value store can be used to
hash keys to an unique destination node on traditional distributed systems
\cite{chord} \cite{dynamo}. By using a ring-structured distributed hash table,
high scalability is ensured by establishing an upper bound on the number of keys
that each node is responsible for. In addition, the load across servers can be
balanced by distributing the keys as evenly as possible across nodes on
different clusters, and even across data centers\cite{dynamo}.

The ring-based replication policy also makes the data highly available by
maximizing the number of nodes that contain replicas. However, for this paper we
will not consider availability, since data will be accessed by a single
coordinator client. Our Device Mapper Cache implementation uses a key-value
replication approach that is very similar to that found in \cite{chord} in order
to cooperatively replicate the clean pages that are used least often, and
invalidate without having read from disk again later. The objective is to
maximize the working set across SSD caches in the storage network and give
storage server HDD's more chances to remain spun down for a long period of time.

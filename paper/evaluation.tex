\section{Evaluation}
\label{sec:evaluation}

In this section we present the results of evaluating the energy-saving effects
of using a block-level solution on a SAN environment. There are two
configurations that are compared; the first is the baseline case where a client
machine accesses its back-end storage through an iSCSI connection to a remoter
storage server. In this case, both client and server machines are utilized when
I/O is performed. The client's memory capacity will determine the amount of data
that is cached from the server-side and thus the amount of interactions with the
server. As for the DM-Cache setup, an SSD is used an as additional cache layer
between the client and the server. This layer provides a larger capacity to
cache more data from the server-side, and thus reduce interactions with it. Two
kinds of configurations of DM-Cache are tested; one that caches data from read
operations but sends writes directly to the source device, and one configuration
that caches data from both read and write operations.

\subsection{Experimental Setup}

These tests were performed on two nodes that are part of a Dell 2970
cluster. The source device and the cache device were in separate servers, which
are running Linux 2.6.32 on Ubuntu with 24 GB of RAM.  The initiator (client) is
connected to the target device (source device) using the iSCSI protocol. The
initiator server contained the two disks, an HDD that was used as local storage,
and an SSD that was used as the cache device. The HDD was a 1 TB Seagate
Constellation ES 7200 RPM disk. The SSD cache was a 120 GB Intel SSD 320
Series. The target server (storage server) contained four 1TB HDD with
specifications similar to the one used in the initiator server. One of these was
used for local storage, and another was used as the source device. The remaining
two were not used as part of the tests. For every test DM-Cache was configured
to use a block size of 8 sectors, an associativity of 256, and a cache size of
64 GB. As for the benchmarking tool, IOzone 3.398 was used. All I/O operations
were done on a file of size of 8GB or 12GB, and with a request size of 4 KB.

\subsection{Measuring Power Consumption}

In order to measure the power consumption of a node we used the Watts up? PRO
power meter \cite{wattsup}. This power meter measures the power consumption in
watts with an interval as fast as one second. It came with a USB cable adapter
which we used in combination with an open-source Linux utility tool in order to
download the data directly to a PC.

\subsection{Workloads}

There are several workloads that can receive potential power consumption saving
from using client-side caching. Specifically, clients will benefit more when the
workloads access data that has already been cached to the client-side SSD
device. A system that uses dm-cache with writeback configuration can support
operations like these. For example, write and read operations which produce hits
on the cache will reduce load on the server-side storage device, and at the same
time reduce the amount of power consumed on the storage server. For the baseline
system setup, a read may be served from the system’s page cache, but writes will
always have to be dispatched to the storage server. This results in a constant
use of the server storage device.

Because of the nature of a cache device, we also consider the effect of cache
misses on power consumption. Cache misses caused by operations like a cold read
will inevitably have greater power consumption than the current iSCSI
system. This is because they result in a read operation to read data from the
server storage, and a write operation to write data on the client-side SSD
cache. Thus, workloads that have greater cache hit ratios are the most
appropriate for this type of system.

\subsection{Energy Consumption Results}

\begin{table}
  \centering
  \resizebox{\linewidth}{!}
  {
    \begin{tabular}{|l|l|l|l|l|}
      \hline & \bf iSCSI-W & \bf DMC-WT & \bf DMC-WB-M & \bf DMC-WB-H \\ \hline
      Client & 15589.3     & 16035.6    & 13316.2      & 11664.2      \\ \hline
      Server & 17754.7     & 17874.1    & 14479.4      & 13170.5      \\ \hline
      Total  & 33344       & 33909.7    & 27795.6      & 24834.7      \\ \hline
    \end{tabular}
  }
  \caption{Energy consumption (in Joules) of write operations using an 8GB file}
  \label{tab:write-energy}
\end{table}

\begin{table}
  \centering
  \resizebox{\linewidth}{!}
  {
    \begin{tabular}{|l|l|l|l|}
      \hline & \bf iSCSI-R & \bf DMC-R-M & \bf DMC-R-H \\ \hline
      Client & 13168.8     & 16163.8     & 9119.3      \\ \hline
      Server & 15189       & 18068.1     & 10363.8     \\ \hline
      Total  & 28537.8     & 34231.9     & 19483.1     \\ \hline
    \end{tabular}
  }
  \caption{Energy consumption (in Joules) of read operations using an 8GB file}
  \label{tab:read-energy}
\end{table}